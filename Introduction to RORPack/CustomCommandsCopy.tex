%\ProvidesPackage{CustomCommands}

\usepackage{ifthen}
\newcommand*{\keyterm}[1]{\emph{#1}}% Math Operators
\DeclareMathOperator{\re}{Re}			% Real Part
\DeclareMathOperator{\im}{Im}			% Imaginary Part
\DeclareMathOperator*{\Res}{Res}		% Residue
\DeclareMathOperator{\rank}{rank}		% Rank
\DeclareMathOperator{\diag}{diag}		% Diagonal
\DeclareMathOperator{\Tr}{Tr}			% Trace
\DeclareMathOperator{\adj}{adj}		% Classical Adjoint
\DeclareMathOperator{\spr}{spr}		% Spectral Radius
\DeclareMathOperator{\Span}{span}		% Span
\DeclareMathOperator{\Gr}{Gr}			% Graph of a mapping
\DeclareMathOperator*{\lcm}{lcm}         % Least Common Multiple
\DeclareMathOperator{\supp}{supp}        % Support
%\DeclareMathOperator{\ran}{ran}          % Range
\newcommand{\ran}{\mathcal{R}}						% Range
\renewcommand{\ker}{\mathcal{N}}					% Null space
\DeclareMathOperator*{\essup}{ess\,sup}  % Essential Supremum
\DeclareMathOperator*{\vrai}{vrai}
\DeclareMathOperator*{\wlim}{w-lim}      % Weak Limit
\DeclareMathOperator*{\slim}{s-lim}      % Strong Limit
\DeclareMathOperator*{\ulim}{u-lim}      % Uniform Limit
\DeclareMathOperator*{\dist}{dist}       % Distance between sets
\DeclareMathOperator*{\diam}{diam}       % Diameter of a set
\DeclareMathOperator*{\card}{card}       % Cardinality of a set
\DeclareMathOperator{\codim}{codim}			 % Co-dimension
\DeclareMathOperator{\alg}{alg}					% Algebraic multiplicity
\DeclareMathOperator{\geom}{geom}					% Geometric multiplicity
\DeclareMathOperator{\sinc}{sinc}
%\DeclareMathOperator{\Arg}{Arg} 	% Principal argument

% Elementary functions
% Names starting with capital letters are principal branches.
\DeclareMathOperator{\Arg}{Arg}
\DeclareMathOperator{\Log}{Log}
\DeclareMathOperator{\Ln}{Ln}
\DeclareMathOperator{\arccot}{arccot}
\DeclareMathOperator{\Arcsin}{Arcsin}
\DeclareMathOperator{\Arccos}{Arccos}
\DeclareMathOperator{\Arctan}{Arctan}
\DeclareMathOperator{\Arccot}{Arccot}
\DeclareMathOperator{\arcsinh}{arcsinh}
\DeclareMathOperator{\arccosh}{arccosh}
\DeclareMathOperator{\arctanh}{arctanh}
\DeclareMathOperator{\arccoth}{arccoth}
\DeclareMathOperator{\Arcsinh}{Arcsinh}
\DeclareMathOperator{\Arccosh}{Arccosh}
\DeclareMathOperator{\Arctanh}{Arctanh}
\DeclareMathOperator{\Arccoth}{Arccoth}

\DeclareMathOperator{\sign}{sign}

% Direct and orthogonal sums
\newcommand{\dirsum}{\oplus}
\newcommand{\orthsum}{\oplus_{\hspace{-1pt}\raisebox{-1pt}{\mbox{\scriptsize$\bm{\perp}$}}\hspace{-2pt}}}


% Variable size delimiters.
% The first argument is the size: big, bigg, etc.
% The second argument is the delimiter: \{, [, (, etc.
\newcommand*{\ldelim}[2]{\csname#1l\endcsname#2}   % Left
\newcommand*{\rdelim}[2]{\csname#1r\endcsname#2}   % Right
\newcommand*{\mdelim}[2]{\csname#1m\endcsname#2}   % Middle

% Horizontal space the amount of a relational symbol
\newcommand*{\relspace}[1][=]{\mathrel{\phantom{#1}}}
% Raise a superscript
\newcommand*{\raisesp}[2]{\raisebox{#1}{\mbox{$\scriptstyle#2$}}}

% Greek letters
\newcommand{\ga}{\alpha}
\newcommand{\gb}{\beta}
\newcommand{\gc}{\chi}
\renewcommand{\gg}{\gamma}
\newcommand{\gd}{\delta}
\newcommand{\gf}{\varphi}
\newcommand{\gl}{\lambda}
\newcommand{\gw}{\omega}
\newcommand{\gs}{\sigma}
\newcommand{\gth}{\theta}
\newcommand{\gt}{\tau}
\newcommand{\gz}{\zeta}
\newcommand{\eps}{\varepsilon}

% Symbols
\newcommand*{\C}{{\mathbb{C}}}     % Complex
\newcommand*{\Cp}{{\mathbb{C}^+}}
\newcommand*{\R}{{\mathbb{R}}}     % Real
\newcommand*{\Q}{{\mathbb{Q}}}     % Rational
\newcommand*{\Z}{{\mathbb{Z}}}     % Integer
\newcommand*{\N}{{\mathbb{N}}}     % Natural

% Upright letters

\newcommand*{\me}{{\mathrm{e}}}
\newcommand*{\mi}{{\mathrm{i}}}
\newcommand*{\md}{{\mathrm{d}}}
\newcommand*{\mD}{{\mathrm{D}}}
\newcommand*{\minsv}{\sigma_{\textnormal{min}}}   % Minimum singular value
\newcommand*{\maxsv}{\sigma_{\textnormal{max}}}   % Maximum singular value
\newcommand*{\Lin}{{\mathcal{L}}}   % Bounded Linear Operators
\newcommand*{\Clo}{\mathcal{C}}     % Closed Linear Operators
\newcommand*{\Comp}{\mathcal{K}}		% Compact Linear Operators
\newcommand*{\Dom}{{\mathcal{D}}}   % Domain
\newcommand*{\Ran}{{\mathcal{R}}}   % Range
\newcommand*{\Nul}{{\mathcal{N}}}   % Nullspace
\newcommand*{\Contr}{{\mathcal{B}}} % Controllability map
\newcommand*{\Obser}{{\mathcal{C}}} % Observability map

% Fixed size absolute value etc.
\newcommand*{\abs} [1]{\lvert#1\rvert}
\newcommand*{\norm}[1]{\lVert#1\rVert}
\newcommand*{\seminorm}[1]{\abs{#1}}
\newcommand*{\set} [1]{\{#1\}}
\newcommand*{\setm}[2]{\{\,#1\mid#2\,\}}   % Set with middle |
\newcommand*{\iprod}[2]{\langle#1,#2\rangle}    % Inner Product
\newcommand*{\dualpair}[2]{\langle#1,#2\rangle}
\newcommand*{\pair}[2]{\left(#1,#2\right)} % Ordered pair
\newcommand*{\conv}[2]{#1\ast#2}
\newcommand*{\seq}[2]{(#1)_{#2}^\infty}  % Sequence
\newcommand*{\seqn}[2][]{(#2)_{#1}}      % Sequence
\newcommand*{\ceil} [1]{\lceil#1\rceil}
\newcommand*{\floor}[1]{\lfloor#1\rfloor}
\newcommand*{\bin}[2]{\left(\mat{#1\\#2}\right)}

% Variable size absolute value etc.
% The optional first argument is the size: big, bigg, etc.
% The default is to use automatic sizing with \left and \right.
\newcommand*{\Abs}[2][default]{\ifthenelse{\equal{#1}{default}}{\left\lvert#2\right\rvert}{\ldelim{#1}{\lvert}#2\rdelim{#1}{\rvert}}}
\newcommand*{\Norm}[2][default]{\ifthenelse{\equal{#1}{default}}{\left\lVert#2\right\rVert}{\ldelim{#1}{\lVert}#2\rdelim{#1}{\rVert}}}
\newcommand*{\Seminorm}[2][default]{\Abs[#1]{#2}}
\newcommand*{\Iprod}[3][default]{\ifthenelse{\equal{#1}{default}}{\left\langle#2,#3\right\rangle}{\ldelim{#1}{\langle}#2,#3\rdelim{#1}{\rangle}}}
\newcommand*{\Dualpair}[3][default]{\ifthenelse{\equal{#1}{default}}{\left\langle#2,#3\right\rangle}{\ldelim{#1}{\langle}#2,#3\rdelim{#1}{\rangle}}}
\newcommand*{\Seq}[2]{\left(#1\right)_{#2}^\infty}
\newcommand*{\Seqn}[2][]{\left(#2\right)_{#1}}
\newcommand*{\List}[2][1]{\set{#1,\ldots,#2}}

% Set
% Variable size braces.
% The optional first argument is the size: big, bigg, etc.% The default is to use automatic sizing with \left and \right.
\newcommand*{\Set}[2][default]{\ifthenelse{\equal{#1}{default}}{\left\{#2\right\}}{\ldelim{#1}{\{}#2\rdelim{#1}{\}}}}

% Set with middle
% Variable size braces and |.
% The optional first argument is the size: big, bigg, etc.
% The default is big.
\newcommand*{\Setm}[3][big]{\ldelim{#1}{\{}\,#2\mdelim{#1}{|}#3\,\rdelim{#1}{\}}}

% Partial Derivative
%\newcommand*{\dd}[1][x]{\frac{d}{d#1}}
\newcommand*{\dda}[3][1]{\ifthenelse{\equal{#1}{1}}{\frac{d#3}{d#2}}{\frac{d^{#1}#3}{d#2^{#1}}}}
\newcommand*{\ddb}[2][1]{\ifthenelse{\equal{#1}{1}}{\frac{d}{d#2}}{\frac{d^{#1}}{d#2^{#1}}}}
\newcommand*{\pd}[3][1]{\ifthenelse{\equal{#1}{1}}{\frac{\partial{#2}}{\partial{#3}}}{\frac{\partial^{#1}{#2}}{\partial#3^{#1}}}}
\newcommand*{\pdb}[2][1]{\ifthenelse{\equal{#1}{1}}{\frac{\partial}{\partial{#2}}}{\frac{\partial^{#1}}{\partial#2^{#1}}}}
\newcommand*{\pdd}[3]{\frac{\partial^2{#1}}{\partial{#2}\partial{#3}}}
% Substitution (in integral)
\newcommand{\subst}{\bigg\lvert}
% Matrix
\newcommand{\Matrix}[1]{\begin{bmatrix}#1\end{bmatrix}}
\newcommand{\Det}[1]{\begin{vmatrix}#1\end{vmatrix}}
\newcommand{\Array}[2]{\left[\begin{array}{#1}#2\end{array}\right]}

% Complex Conjugate, Closure, Boundary, Complement
\newcommand*{\conj}[1]{\overline{#1}}
\newcommand*{\closure}[1]{\overline{#1}}
\newcommand{\boundary}{\partial}
\newcommand*{\Complement}{\mathord{\thicksim}}

% Limits from the left and right
\newcommand{\lto}{\uparrow}
\newcommand{\rto}{\downarrow}
\newcommand*{\limn}[1][n]{\lim_{#1\to\infty}}

% Arrows (abbreviations)
%\newcommand{\to}{\rightarrow}
\newcommand{\from}{\leftarrow}
\newcommand{\too}{\longrightarrow}
\newcommand{\tofrom}{\leftrightarrow}
\newcommand{\froom}{\longleftarrow}
\newcommand{\To}{\Rightarrow}
\newcommand{\From}{\Leftarrow}

% Algebra
\newcommand*{\semigroup}[1]{\mathcal{#1}}
\newcommand*{\monoid}[1]{\mathcal{#1}}
\newcommand*{\group}[1]{\mathcal{#1}}
\newcommand*{\ring}[1]{\mathcal{#1}}
\newcommand*{\field}[1]{\mathbb{#1}}
\newcommand*{\divides} [2]{{#1}|{#2}}
\newcommand*{\rdivides}[2]{{#1}|_R\,{#2}}
\newcommand*{\ldivides}[2]{{#1}|_L\,{#2}}

% Misc
\newcommand*{\Xloc}[1]{#1_{\text{loc}}}
\newcommand*{\Lploc}[1][p]{\Xloc{L^{#1}}}
\newcommand*{\Lp}[1][p]{L^{#1}}
\newcommand*{\lp}[1][p]{\ell^{#1}}
\newcommand*{\half}[1][2]{\frac{1}{#1}}
\newcommand*{\range}[3][i]{{#2}\le{#1}\le{#3}}
\newcommand*{\cartesian}{\text{\Large $\times$}}
\newcommand*{\inverse}{^{-1}}
\newcommand*{\inv}{^{-1}}
\newcommand{\pinv}{^{\dagger}}


%\renewcommand{\vec}[1]{\mathbf{#1}}
\renewcommand{\vec}[1]{\underline{#1}}

% Referencing Sections, Theorems etc.
\newcommand*{\refCh}[1]{Chapter~\ref{#1}}        % Chapter
\newcommand*{\refch}[1]{chapter~\ref{#1}}        % lower case
\newcommand*{\refSec}[1]{Section~\ref{#1}}       % Section
\newcommand*{\refsec}[1]{section~\ref{#1}}       % 
\newcommand*{\refFig}[1]{Figure~\ref{#1}}        % Figure
\newcommand*{\reffig}[1]{figure~\ref{#1}}        % 
\newcommand*{\refTbl}[1]{Table~\ref{#1}}         % Table
\newcommand*{\refThm}[1]{Theorem~\ref{#1}}       % Theorem
\newcommand*{\refthm}[1]{theorem~\ref{#1}}       % 
\newcommand*{\refPrp}[1]{Proposition~\ref{#1}}   % Proposition
\newcommand*{\refprp}[1]{proposition~\ref{#1}}   % 
\newcommand*{\refLem}[1]{Lemma~\ref{#1}}         % Lemma
\newcommand*{\reflem}[1]{lemma~\ref{#1}}         % 
\newcommand*{\refCor}[1]{Corollary~\ref{#1}}     % Corollary
\newcommand*{\refcor}[1]{corollary~\ref{#1}}     % 
\newcommand*{\refDef}[1]{Definition~\ref{#1}}    % Definition
\newcommand*{\refdef}[1]{definition~\ref{#1}}    % 
\newcommand*{\refRem}[1]{Remark~\ref{#1}}	 % Remark			 % Remark
\newcommand*{\refrem}[1]{remark~\ref{#1}}	 %			 % 
\newcommand*{\refPro}[1]{Problem~\ref{#1}}       % Problem
\newcommand*{\refExa}[1]{Example~\ref{#1}}       % Example
\newcommand*{\refexa}[1]{example~\ref{#1}}       % 
\newcommand*{\refExe}[1]{Exercise~\ref{#1}}      % Exercise
\newcommand*{\refEq}[1]{Equation~\eqref{#1}}     % Equation
%\newcommand*{\refeq}[1]{equation~\eqref{#1}}     % 
\newcommand*{\refEqs}[1]{Equations~\eqref{#1}}   % Equations
\newcommand*{\refeqs}[1]{equations~\eqref{#1}}   % 
\newcommand*{\refAss}[1]{Assumption~\ref{#1}}    % Assumption
\newcommand*{\refass}[1]{assumption~\ref{#1}}    % 

\DeclareMathOperator*{\AP}{AP}	% Almost periodic functions
\DeclareMathOperator*{\BUC}{BUC}	% BUC functions

\newcommand{\bmat}[1]{\begin{bmatrix}#1\end{bmatrix}}
\newcommand{\bmatsmall}[1]{\begin{bsmallmatrix}#1\end{bsmallmatrix}}
\newcommand{\pmat}[1]{\begin{pmatrix}#1\end{pmatrix}}
\newcommand{\pmatsmall}[1]{\begin{psmallmatrix}#1\end{psmallmatrix}}
\newcommand{\mat}[1]{\begin{matrix}#1\end{matrix}}
\newcommand{\mresult}[1]{\left\{\begin{array}{ll}#1\end{array}\right.}
\newcommand{\mc}[1]{\mathcal{#1}}
\newcommand{\wt}[1]{\widetilde{#1}}
\newcommand{\Csg}{$C_0$-semigroup}
\newcommand{\Cg}{$C_0$-group}


\newcommand{\eq}[1]{\begin{align*}#1\end{align*}}
\newcommand{\eqn}[1]{\begin{align}#1\end{align}}
\newcommand{\ieq}[1]{$#1$}

% Title for notes
\newcommand{\owntitle}[2][\today]{
\clearpage
\begin{center}
{\LARGE #2 \par}
\end{center}
\centerline{\large #1}
}

% Title for notes with author (\owntitleauth[date]{author}{title})
\newcommand{\owntitleauth}[3][\today]{
\clearpage
\begin{center}
{\LARGE #3 \par}
\bigskip
{\large #2}
\end{center}
\centerline{\large #1}
}

% Hide proofs. 
% Usage:
%\newboolean{hideproof}
%\setboolean{hideproof}{true}
%\begin{proof}\hiddenproof{\boolean{hideproof}}{[proof...]}\end{proof}

\newcommand{\hiddenproof}[2]{
\ifthenelse{#1}{Hidden}{#2}
}

% Cite with a location or a theorem number
\newcommand{\citel}[2]{\cite[#2]{#1}}

\newcommand{\longcomm}[1]{}

% Color macros
\newcommand{\blue}[1]{{\color{blue}#1}}
\newcommand{\red}[1]{{\color{red}#1}}
\newcommand{\purple}[1]{{\color{purple}#1}}
\newcommand{\green}[1]{{\color{green}#1}}


\usepackage{datetime}
\newdateformat{findate}{\THEDAY.\THEMONTH.\THEYEAR}

\endinput 
